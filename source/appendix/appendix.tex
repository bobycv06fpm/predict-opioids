\documentclass[11pt, letter]{article}
\usepackage[left=1in, right=1in, top=1in, bottom=1in]{geometry}
\usepackage{graphicx}

% Helvetica font
\usepackage[scaled]{helvet}
\renewcommand\familydefault{\sfdefault} 
\usepackage[T1]{fontenc}
\usepackage{textcomp}

% Hyperlinks
\usepackage{hyperref}
\hypersetup{
    bookmarks=true,
    urlcolor=blue,
    colorlinks=true
}

% Endnotes
\usepackage{endnotes}
\renewcommand{\makeenmark}{\textsuperscript{\theenmark}}
\makeatletter
\def\enoteformat{\textsuperscript{\theenmark}}
\makeatother

% Paragraphs
\usepackage[parfill]{parskip}

% Import references
\usepackage{hyperref}

\newcommand{\refashp}{
{\em AHFS\textsuperscript \textregistered Pharmacologic/Therapeutic Classification\textsuperscript \textcopyright} used with permission. \textcopyright\ 2017, the American Society of Health-System Pharmacists, Inc. (ASHP). The Data is a part of the {\em AHFS Drug Information\textsuperscript \textregistered}; ASHP is not responsible for the accuracy of transpositions from the original context.
}

\newcommand{\refndc}{
U.S. Food \& Drug Administration.
National Drug Code Directory [Internet; cited 2019 Feb 19];
Available from: \url{https://www.fda.gov/Drugs/InformationOnDrugs/ucm142438.htm}
}

\newcommand{\refhistoricalndc}{
Research Improving People's Lives.
Assembling a Historical National Drug Code Directory from the Internet Archive [Internet; cited 2019 Feb 19];
Available from: \url{https://github.com/ripl-org/historical-ndc}
}

\newcommand{\refamdg}{
Agency Medical Directors' Group. Interagency Guideline on Prescribing Opioids for Pain.
Olympia, WA: 2015 [cited 2019 Feb 19].
Available from: \url{http://www.agencymeddirectors.wa.gov/Files/2015AMDGOpioidGuideline.pdf}
}

\newcommand{\refdowell}{
Dowell D, Haegerich TM, Chou R. CDC Guideline for Prescribing Opioids for Chronic Pain — United States, 2016. MMWR Recomm Rep [Internet]. 2016 [cited 2019 Feb 19];65. Available from: \url{https://www.cdc.gov/mmwr/volumes/65/rr/rr6501e1.htm}
}

\newcommand{\refrice}{
Rice JB, White AG, Birnbaum HG, Schiller M, Brown DA, Roland CL. A model to identify patients at risk for prescription opioid abuse, dependence, and misuse. Pain Med. 2012;13(9):1162–1173. 
}

\newcommand{\refhylan}{
Hylan TR, Von Korff M, Saunders K, Masters E, Palmer RE, Carrell D, et al. Automated Prediction of Risk for Problem Opioid Use in a Primary Care Setting. J Pain. 2015 Apr;16(4):380–7. 
}

\newcommand{\refdufour}{
Dufour R, Mardekian J, Pasquale MK, Schaaf D, Andrews GA, Patel NC. Understanding predictors of opioid abuse: predictive model development and validation. Am J Pharm Benefits. 2014;6(5):208–216. 
}

\newcommand{\refjones}{
Jones CM. Heroin use and heroin use risk behaviors among nonmedical users of prescription opioid pain relievers – United States, 2002–2004 and 2008–2010. Drug Alcohol Depend. 2013 Sep;132(1–2):95–100. 
}

\newcommand{\refcarrell}{
Carrell DS, Cronkite D, Palmer RE, Saunders K, Gross DE, Masters ET, et al. Using natural language processing to identify problem usage of prescription opioids. Int J Med Inform. 2015 Dec;84(12):1057–64. 
}

\newcommand{\refabadi}{
Abadi M, Barham P, Chen J, Chen Z, Davis A, Dean J, et al. TensorFlow: A System for Large-Scale Machine Learning. In: 12th {USENIX} Symposium on Operating Systems Design and Implementation ({OSDI} 16). 2016 [cited 2019 Feb 19]. p. 265–83. Available from: \url{https://www.usenix.org/conference/osdi16/technical-sessions/presentation/abadi}
}

\newcommand{\refkingma}{
Kingma DP, Ba J. Adam: A Method for Stochastic Optimization. arXiv:14126980 [Internet]. 2014 Dec 22 [cited 2019 Feb 19]; Available from: \url{http://arxiv.org/abs/1412.6980}
}

\newcommand{\refsrivastava}{
Srivastava N, Hinton G, Krizhevsky A, Sutskever I, Salakhutdinov R. Dropout: A Simple Way to Prevent Neural Networks from Overfitting. J Mach Learn Res. 2014 Jun 14;15:1929–58. 
}

\newcomand{\refbarocas}{
Barocas JA, White LF, Wang J, Walley AY, LaRochelle MR, Bernson D, et al. Estimated Prevalence of Opioid Use Disorder in Massachusetts, 2011–2015: A Capture–Recapture Analysis. Am J Public Health. 2018 Oct 25;108(12):1675–81. 
}

\newcommand{\refhastings}{
Hastings JS, Howison M, Lawless T, Ucles J, White P. Integrating Administrative Data for Policy Insights. OSF Preprints [Internet]. 2018 Jul 19 [cited 2018 Aug 23]; Available from: \url{https://osf.io/hdv3c/}
}

\newcommand{\refhochreiter}{
Hochreiter S, Schmidhuber J. Long Short-Term Memory. Neural Computation. 1997 Nov 1;9(8):1735–80. 
}




\begin{document}

\begin{center}
\huge Appendix
\end{center}

\section{Experimental Design}

Our objective was to define a panel of Rhode Island Medicaid recipients who received an initial opioid prescription under Medicaid coverage; define adverse outcomes of opioid dependence, abuse, or poisoning; and model and assess the accuracy of predictions of adverse outcomes using information known only prior to the initial prescription. Data were split into randomly-sampled training, validation, and testing sets using the ratio 50:25:25 at the beginning of the study. We report the results of model predictions on the testing set, which was withheld from analysis prior to the preparation of the manuscript.

Data are from the period 2001-2017, and include Rhode Island administrative records from the Department of Human Services (DHS), Department of Labor and Training (DLT), Department of Corrections (DOC), Medicaid program (under the Executive Office of Health and Human Services), and police agencies (including the Rhode Island State Police and eight municipal police departments).

Although our data span the years 2001 to 2017, we construct a panel of individuals with initial prescriptions between 2006 and 2012 to allow for the construction of features up to five years before the initial prescription and to define outcomes up to five years after the initial prescription.

\section{Opioid Prescriptions}

To define our panel, we first establish which pharmacy claims correspond to opioid prescriptions. The primary identifier for the dispensed drug is a standardized 11-digit National Drug Code (NDC) from the U.S. Food and Drug Administration's NDC Directory.\endnote{\refndc} This directory is only available as a current snapshot, and because our claims data start in 2001, there are many unmapped NDCs to the current directory. Out of approximately 14.8 million pharmacy claims between 2006 and 2012, only 66.8 percent join to the current directory. Therefore, we construct a historical NDC directory using a data mining framework that downloads and collates all available Internet Archive snapshots of the FDA's NDC website since 2000.\endnote{\refhistoricalndc} This historical directory also includes full ingredient lists for each NDC, standardized to milligrams. Using this improved directory, 88.1 percent of pharmacy claims between 2006 and 2012 map to an NDC entry. 

We define an opioid prescription as any claim for a drug containing an opioid ingredient at or above the recommended starting dose when initiating opioid therapy for chronic pain management, as established in Washington State's 2015 prescribing guideline and further cited in the Centers for Disease Control's 2016 prescribing guideline.\endnote{\refamdg}\textsuperscript{,}\endnote{\refdowell} Table A1 summarizes these ingredients and the minimum amounts we use to define an opioid drug. Of the 4,359 drugs containing one of these ingredients, 4,175 meet the minimum threshold amount and appear in 3.9 percent of claims.

Additionally, we define a recovery prescription as any NDC containing one of four ingredients commonly used in medication-assisted treatment of an opioid use disorder, which identifies 412 such drugs that appear in 0.5 percent of claims. These prescriptions may indicate that an individual has a preexisting opioid use disorder.

\section{Outcomes}

For each individual in our panel, we examine all of the Medicaid claims following their initial opioid prescription to construct indicator variables for four types of adverse outcomes: opioid dependence, opioid abuse, prescription-opioid poisoning and heroin poisoning. We include heroin poisoning as an outcome given the increasing use of heroin among those who abuse opioids, and the high proportion (greater than 80 percent) of joint heroin-prescription-opioid users who abused opioids prior to using heroin.\endnote{\refjones}

We determine these outcomes from the claim's International Classification of Diseases (ICD) diagnosis codes, which are used by medical professionals to classify a patient's health conditions following an encounter. Because our data span the transition from the ICD-9 to ICD-10 classification, we include diagnosis codes from both. Table A2 summarizes the exact codes used to indicate each of these four diagnosis-related outcomes. 

Not everyone with an opioid use disorder receives a diagnosis code. Though it is unknown precisely what fraction of opioid use disorders go undiagnosed, Carrell {\em et al}. found that diagnosis codes were missing for as many as a quarter of patients for whom their providers were aware of opioid abuse.\endnote{\refcarrell} Similarly, a study by Barocas {\em et al}. estimated that only 44\% of individuals with opioid use disorder were identified as such in claims and administrative records.\endnote{\refbarocas} To address the challenges with diagnoses codes, we define a fifth treatment outcome using procedure codes related to the treatment of opioid use disorder, and more generally for drug rehabilitation and detoxification (Table A2). Finally, we define a sixth ``any" outcome as the union of any of the diagnoses or treatment outcomes, to capture as broad a population of individual with opioid use disorder as possible. Data and measurement limitations notwithstanding, our model demonstrates that administrative data can be combined to form an accurate prediction of these outcomes, suggesting a feasible path forward for utilizing data to inform prescription risk.

\section{Final Panel}

Out of 400,024 distinct Medicaid enrollees between 2006 and 2012, our panel initially contains 74,213 individuals who received at least one opioid prescription in that period. We exclude 511 individuals who received a recovery prescription before their initial opioid prescription, since this indicates they may have been seeking treatment for an opioid use disorder. We exclude 3,549 individuals with an adverse outcome prior to their initial opioid prescription, since we assume they were already receiving opioids from another source, such as through private insurance before enrolling in Medicaid. Our final panel includes 70,153 individuals. Table A3 shows the incidence of adverse outcomes among these individuals, by baseline characteristics from the RI360\endnote{\refhastings} database.

\section{Feature Engineering}

\section{Tensor Representation}

We construct tensors of monthly values for a given feature for each of the individuals in our panel in the 12 months prior to the individual's initial prescription. Missing values are imputed using mean values from the training population.

The DHS tensor includes 13 features for demographics (age and indicators for sex, race, and Spanish or Portuguese as a primary language) and monthly payments for the Supplemental Nutrition Assistance Program (SNAP), the Temporary Assistance for Needy Families (TANF), the General Public Assistance (GPA), the Child Care Assistance Program (CCAP), and State Supplemental Payment portions of Supplemental Security Income benefits.

The DLT tensor includes 31 features for indicators for sector of work derived from the first two digits of industry codes assigned according to the North American Industry Classification System (NAICS); monthly payments for Temporary Disability Insurance (TDI) and Unemployment Insurance (UI); and quarterly wage history, including wage amount, the number of employers and the number of hours worked (for hourly employees).

The DOC tensor includes 16 features for demographics (age and indicators for sex, race, Spanish as a primary language), and indicators for charges, seven categories of sentencing, and commitments and releases from prison.

The Medicaid tensor includes 928 features for demographics (age and indicators for sex, race, and Spanish or Portuguese as a primary language); indicators for eligibility categories, plan type, and payer codes at each month of enrollment; number of claims and total bill and payment amounts for all claims and for Emergency Department claims; the number of prescriptions in each of the 141 second-level categories from the AHFS Pharmacologic/Therapeutic Classification;\endnote{\refashp} and indicators for ICD-9 diagnoses codes and HCPCS procedure codes for all codes that are correlated >0.02 with any adverse outcome in the training population.

The police tensor includes 42 features for demographics (age and indicators for sex and officer-observed race); indicators for all arrests, DUI arrests, and domestic-offense arrests; the number of car crashes involved and injured in; the number of and total fines for citations; and the spatio-temporal intensity of calls for service in the individual's home Census block group for 29 categories of calls.

Finally, we construct an integrated tensor including all of the 1,030 features from the DHS, DLT, DOC, Medicaid, and police tensors. The dimension of this integrated tensor are 70,153 individuals x 12 months x 1,030 features.

\section{Models}

\subsection{Linear}

\subsection{Neural Networks}

To obtain predictions of adverse outcomes after the initial prescription, we train a neural network model for each tensor using the Python package Keras (https://keras.io), which provides an interface to the TensorFlow\endnote{\refabadi} library. We specifically choose to train a recurrent neural network (RNN), since RNNs have the ability to model temporal patterns in the input data. We input our training data into a two-layer network of 1x150 Long Short-Term Memory (LSTM)\endnote{\refhochreiter} units and use the tanh activation function. We input last LSTM layer into a dense layer that applies a sigmoid activation function to the weighted sum of the 150 inputs in order to produce a single predicted probability of adverse outcome. We employ regularization prior to each layer in the form of a dropout factor of 0.25, which causes a random deactivation of units within the layer during training with a fixed probability of 0.25.\endnote{\refsrivastava}

The neural networks are optimized to minimize the binary cross-entropy, also known as log-loss, on the training data. We use the Adam\endnote{\refkingma} optimization algorithm, training with a batch size of 32. We tune the model on a hold-out validation set by allowing the neural network to train for as many epochs as needed until the area under the receiver-operator curve (AUC) from predictions on the validation set do not improve by 0.001. In Table A6, we report the AUC from predictions on the testing set for each data source and each individual outcome.

\theendnotes

\newpage

\textbf{Table A1.} Minimum amounts of ingredients in a drug to classify it as an opioid prescription or a recovery prescription.

\begin{tabular}{lc}
\em Opioid Ingredient & \em Minimum Amount (mg) \\[0.5em]
Codeine & 30.0 \\
Fentanyl & 0.0125 \\
Hydrocodone & 5.0 \\
Hydromorphone & 2.0 \\
Meperidine* & 0.0 \\
Morphine & 10.0 \\
Oxycodone & 5.0 \\
Oxymorphone & 5.0 \\
Tapentadol & 50.0 \\
Tramadol & 50.0 \\[1em]
\em Recovery Ingredient° & \em Minimum Amount (mg) \\[0.5em]
Buprenorphine & 0.0 \\
Methadone & 0.0 \\
Naloxone & 0.0 \\
Naltrexone & 0.0 \\[1em]
\end{tabular}

\hrule

\footnotesize
* Meperidine has no recommended starting dose for treatment of chronic pain because of its risk for complications in older adults; therefore, we consider any amount as evidence that the drug is an opioid. \\
° We consider any amount of a recovery ingredient as evidence that the drug may have been used to treat a prior opioid use disorder.
\normalsize

\newpage

\textbf{Table A2.} Diagnosis and procedure codes used to indicate adverse outcomes when occurring in any claim after the initial opioid prescription.

\begin{tabular}{lll}
\em Outcome & \em Code & \em Description \\[1em]
\textbf{Opioid} & 304.0 & Opioid type dependence \\
\textbf{Dependence} & 304.7 & Combinations of opioid type drug with any other drug dependence \\
 & F11.2* & Opioid dependence \\[1em]
\textbf{Opioid Abuse} & 305.0 & Nondependent opioid abuse  \\
 & F11.1* & Opioid abuse \\[1em]
\textbf{Prescription-} & 965.00 & Poisoning by opium (alkaloids), unspecified \\
\textbf{Opioid} & 965.02 & Poisoning by methadone \\
\textbf{Poisoning} & 965.09 & Poisoning by other opiates and related narcotics \\
 & 970.1 & Poisoning by opiate antagonists \\
 & E850.1 & Accidental poisoning by methadone \\
 & E850.2 & Accidental poisoning by other opiates and related narcotics \\
 & E935.1 & Methadone causing adverse effects in therapeutic use \\
 & E935.2 & Other opiates and related narcotics causing adverse effects in therapeutic use \\
 & E940.1 & Opiate antagonists causing adverse effects in therapeutic use \\
 & T400* & Poisoning by, adverse effect of and underdosing of opium \\
 & T402* & Poisoning by, adverse effect of and underdosing of other opioids \\
 & T403* & Poisoning by, adverse effect of and underdosing of methadone \\[1em]
\textbf{Heroin} & 965.01 & Poisoning by heroin \\
\textbf{Poisoning} & E850.0 & Accidental poisoning by heroin \\
 & E935.0 & Heroin causing adverse effects in therapeutic use \\
 & T401* & Poisoning by and adverse effects of heroin \\[1em]
\textbf{Treatment} & J2310° & Naloxone HCI Injection, per 1 mg \\
 & J2315° & Naltrexone injection, depot form, 1mg \\
 & J0592° & Buprenorphine HCL injection, 0.1mg \\
 & X0305° & Methadone detoxification – outpatient \\
 & X0321° & Methadone maintenance, assessment and evaluation, counseling, treatment \\
 &  & and review, and lab testing \\
 & H0020° & Alcohol and or drug services; methadone administration and or service \\
 & J1230° & Injection, methadone, up to 10mg \\
 & 83840° & methadone \\
 & 946° & Alcohol and drug rehabilitation and counseling \\
 & 9464° & drug rehabilitation \\
 & 9465° & drug detoxification \\
 & 9466° & drug rehabilitation and detoxification \\
 & 9467° & combined alcohol and drug rehabilitation \\
 & 9468° & combined alcohol and drug detoxification \\
 & 9469° & combined alcohol and drug rehabilitation and detoxification \\[1em]
\end{tabular}

\hrule

\footnotesize
* ICD-10 diagnosis code \\
° HCPCS procedure code
\normalsize

\newpage

\textbf{Table A3.} Descriptive statistics for the final panel.

\input{output/tables/TableA3}

\newpage

\textbf{Table A6.} Area Under the Receiver-Operating Characteristic Curve (AUC) of neural network models using different subsets of administrative data and outcome definitions. Confidence intervals are calculated from 100 bootstrap replicates.

\footnotesize
\input{output/tables/TableA6}
\normalsize

\newpage

\textbf{Figure A1.} Cumulative frequency of adverse outcomes over time since initial opioid prescription. Adverse outcomes are indicated by the diagnosis and procedure codes in Medicaid claims following the initial prescription. An individual may experience multiple types of adverse outcomes, and ``any" is the union of the five specific outcome types. Opioid dependence is the most prevalent of the types.

\includegraphics[width=1\linewidth]{output/figures/FigureA1}

\end{document}
